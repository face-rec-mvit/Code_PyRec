\documentclass[10pt,a4paper]{article}
\begin{document}
\section{INTRODUCTION}
As one of the most important applications of image analysis and understanding,
face recognition has recently received significant attention, especially during the
past 15 years. There are at least two reasons for this trend: the first is the wide range
of commercial and law-enforcement applications, and the second is the availability
of feasible technologies.Though machine recognition of faces has reached a certain level of maturity
after more than 35 years of research, e.g., in terms of the number of subjects to
be recognized, technological challenges still exist in many aspects.As a
result, the problem of face recognition remains attractive to many researchers
across multiple disciplines, including image processing, pattern recognition and
learning, computer vision, computer graphics, neuroscience, and psychology.In a very narrowest sense face recognition means the recognition of facial ID. In a broad sense or from a system
point of view, it implies face detection, feature extraction, and recognition of
facial ID.\\ \\

Although very reliable methods of biometric personal identification exist, e.g.,
fingerprint analysis and retinal or iris scans, these methods rely on the cooperation
of the participants, whereas a personal identification system based on analysis of
frontal or profile images of the face is often effective without the participant’s
cooperation or knowledge.Commercial and law-enforcement applications of face-recognition technology
(FRT) range from static, controlled-format photographs to uncontrolled video
images, posing a wide range of technical challenges and requiring an equally wide
range of techniques from image processing, analysis, understanding, and pattern
recognition.\\ \\
\\ \\ \\ 
PROBLEM STATEMENT: \\

A general statement of the problem of machine recognition of faces can be formulated
as follows: Given still or video images of a scene, identify or verify one
or more persons in the scene using a stored database of faces. Available collateral
information such as race, age, gender, facial expression, or speech may be
used in narrowing the search (enhancing recognition).
\\ \\ \\
The solution to the problem
involves segmentation of faces (face detection) from cluttered scenes, feature
extraction from the face regions, recognition, or verification. In identification problems, the input
to the system is an unknown face, and the system reports back the determined
identity from a database of known individuals, whereas in verification problems,
the system needs to confirm or reject the claimed identity of the input face.
\\ \\ \\
BRIEF HISTORY\\ \\
The earliest work on face recognition can be traced back at least to the 1950s
in psychology [29] and to the 1960s in the engineering literature [20]. (Some of
the earliest studies include work on facial expression of emotions by Darwin [21]
(see also Ekman [30]) and on facial profile based biometrics by Galton [22]). But
research on automatic machine recognition of faces really started in the 1970s
after the seminal work of Kanade [23] and Kelly [24]. Over the past thirty years,
extensive research has been conducted by psychophysicists, neuroscientists, and
engineers on various aspects of face recognition by humans and machines.Psychophysicists and neuroscientists have been concerned with issues such as
whether face perception is a dedicated process and whether it is done holistically or by
local feature analysis.Many of the hypotheses and
theories put forward by researchers in these disciplines have been based on rather
small sets of images. Nevertheless, many of the findings have important consequences
for engineers who design algorithms and systems for machine recognition of human faces.During the early and mid-1970s, typical
pattern-classification techniques, which use measured attributes of features
(e.g., the distances between important points) in faces or face profiles, were
used [20, 24, 23]. During the 1980s, work on face recognition remained largely
dormant. Since the early 1990s, research interest in FRT has grown significantly.
One can attribute this to several reasons: an increase in interest in commercial
opportunities; the availability of real-time hardware; and the emergence of
surveillance-related applications.Over the past 15 years, research has focused on how to make face-recognition
systems fully automatic by tackling problems such as localization of a face in
a given image or a video clip and extraction of features such as eyes, mouth,
etc. Meanwhile, significant advances have been made in the design of classifiers
for successful face recognition. Among appearance-based holistic approaches,
eigen faces [122, 73] and Fisher faces [52, 56, 79] have proved to be effective in
experiments with large databases. Feature-based graph-matching approaches [49]
have also been quite successful. Compared to holistic approaches, feature-based methods are less sensitive to variations in illumination and viewpoint and to inaccuracy
in face localization.During the past five to ten years, much research has been concentrated on video based
face recognition.The still-image problem has several inherent advantages
and disadvantages. For applications such as airport surveillance, automatic location
and segmentation of a face could pose serious challenges to any segmentation
algorithm if only a static picture of a large, crowded area is available. On the other
hand, if a video sequence is available, segmentation of a moving person can be
more easily accomplished using motion as a cue. In addition, a sequence of images
might help to boost the recognition performance if we can effectively utilize all
these images. But the small size and low image quality of faces captured from
video can significantly increase the difficulty in recognition.More recently, significant advances have been made in 3D-based face recognition.Though it is known that face recognition using 3D images has many
advantages over recognition using a single or sequence of 2D images, no serious
effort was made for 3D face recognition until recently. This was mainly due
to the feasibility, complexity, and computational cost for acquiring 3D data in
real time.\\
The Study of the amazing capability of human
perception of faces can shed some light on how to improve existing systems
for machine perception of faces i.e; Neuroscience and psychology.\\
Face perception is an important capability of human perception system and is a
routine task for humans, while building a similar computer system is still a daunting
task. Human recognition processes utilize a broad spectrum of stimuli, obtained from many of the senses.In many situations, contextual knowledge is also applied, e.g., the context plays
an important role in recognizing faces in relation to where they are supposed to be
located.However, the
human brain has its limitations in the total number of persons that it can accurately
remember.A key advantage of a computer system is its capacity to handle large
numbers of face images.\\ \\ \\ 


\subsection{References}
\end{document}