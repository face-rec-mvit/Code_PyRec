\documentclass[10pt,a4paper]{article}
\begin{document}
\newcommand{\tab}{\hspace*{4em}}


\section{A Learning Algorithm for Optimal Face Recognition in Dynamic Environments } 

There have been many different challenges for face recognition like lighting, occlusion, pose varition etc , and theres no algorithm which can handle all the challenges together, different face recognition algorithms handle different challenges to different extent. So we try to propose a new novel algorithm named Adaptive Learning Algirithm, which is similar to simple perceptron learning algorithm in neural networks. 

A Learning Algorithm for Optimal Face Recognition in Dynamic Environments : 

\subsubsection{Pre-requisite :} 

\tab (1)Identification of mapping ( which algorithm suits well for which \tab  database ) \\
\tab (2)Mapping is stored for persistance usage ( Using pickle in python or \tab ObjectSerelization in java ) \\

\subsubsection{Algorithm :}

During the testing two cases are possible :  \\ 

1) The given test database might previously be trained  \\
2) The given test database might be new one ( previously might not be trained ) \\

\textbf{pseudo code of the adaptive learning algorithm : \\}

function checkifgivenistrained(test,train) : \\
begin : \\
\tab  (1) extracting the 16 metrics of the randomly selected image in test db \\
\tab  (2) extracting the 16 metrics of the same [index] randomly selected \tab image in train db \\
\tab \textit{Note: see below for the 16 metrics used} \\
\tab  (3) if (all the metrics match ) \\
\tab \tab  	return (1) \\
end : \\

function findoutsimilarity() : \\
begin : \\
\tab 	(1)Obtain the histogram of the entire test database \\
\tab	(2)Obtain the histogram of the previously trained databases \\
\tab	(3)Compare the test database histgram with the histogram of all the \tab previously trained databases \\
\tab	(4)select the algorithm of the database which ever matches the most \\
\tab	(5)return the algorithm \\
end : \\

ALAmain() : \\
begin: \\
\tab  for each database in previously trained databases : \\
\tab \tab beginfor:\\
\tab \tab   if(checkifgivenistrained): \\
\tab \tab \tab (1) print database identified \\ 
\tab \tab \tab (2) from previously created mapping extract the algo \tab \tab \tab  and run the test database using this algorithm \\
\tab \tab \tab (3) exit ( means executed properly coz db identified \tab \tab \tab from previous trained one ) \\
\tab \tab endfor:\\
\\
\tab if ( testdatabaseisnew ) : \\
\tab begin: \\
\tab \tab (1) algo=findoutsimilarity() \\
\tab \tab (2) run the given test database with the algo (obtained in \tab \tab previous step ) \\
\tab \tab (3) Once algo is identified do the following steps in block \\
\tab \tab begin : \\
\tab \tab \tab (1)Add the entire new database to our trained set.  \\
\tab \tab \tab (2)Update the mapping data file. \\
\tab \tab \tab (3)So that when ever next time same database comes, \tab \tab \tab our algorithm would have learnt which face-recognition \tab \tab \tab algorithm to use. \\
\tab \tab end : \\

\tab end: \\
end: \\    

\end{document}
