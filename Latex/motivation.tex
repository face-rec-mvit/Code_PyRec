\section{Motivation}

Research on face detection followed by recognition is confronted with many challenges. Algorithms must deal with varying illumination, pose, rotation angles, low image quality, occlusion, and background noise in complex real-life situations. The design of face recognition algorithms that can effectively tackle the above said challenges, is an area of ongoing research. Several models have been proposed, from appearance based approaches to sophisticated systems based on Infrared and 3D imaging.Most of the proposed Face Recognition Algorithms deal with a small subset of the challenges mentioned above, here lies the motivation of the proposed project work. In a dynamic scenario (a classroom scene) the lighting keeps varying though the day. People keep moving around in the class. Some lectures are dull , some are enthusiastic which results in different facial expressions of the students. As can be seen, the Face Recognition challenges keep varying in a random fashion in such a dynamic scenario. We propose a learning model which characterises the dynamic scene and identifies the most optimal Recognition algorithm, from a bank of large number of algorithms, which gives the most optimal recognition results for the characteristic (dynamic) scene.